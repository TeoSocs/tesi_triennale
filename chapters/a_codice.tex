\definecolor{mygreen}{rgb}{0,0.6,0}
\definecolor{mygray}{rgb}{0.5,0.5,0.5}
\definecolor{mymauve}{rgb}{0.58,0,0.82}
\definecolor{darkgray}{rgb}{.4,.4,.4}
\definecolor{purple}{rgb}{0.65, 0.12, 0.82}

\lstset{
	basicstyle=\footnotesize \ttfamily,
	commentstyle=\color{mygreen},
	keywordstyle=\color{blue},
	stringstyle=\color{mymauve},
	numbers=none,
	%numberstyle=\tiny\color{lightgray},
	%numbersep=5pt,
	%frame=single,
	rulecolor=\color{mygray}, 
	breakatwhitespace=false,         % sets if automatic breaks should only happen at whitespace
	breaklines=true,                 % sets automatic line breaking
	tabsize=4,
	showspaces=false,
	showstringspaces=false
}

%define Javascript language
\lstdefinelanguage{JavaScript}{
	keywords={typeof, new, true, false, catch, function, return, null, catch, switch, var, if, in, while, do, else, case, break},
	keywordstyle=\color{blue}\bfseries,
	ndkeywords={class, export, boolean, throw, implements, import, this},
	ndkeywordstyle=\color{darkgray}\bfseries,
	identifierstyle=\color{black},
	sensitive=false,
	comment=[l]{//},
	morecomment=[s]{/*}{*/},
	commentstyle=\color{purple}\ttfamily,
	stringstyle=\color{red}\ttfamily,
	morestring=[b]',
	morestring=[b]",
	showspaces=false,
	showstringspaces=false
}

\lstdefinelanguage{docker-compose}{
	keywords={image, environment, ports, container_name, ports, volumes, links},
	keywordstyle=\color{blue}\bfseries,
	identifierstyle=\color{black},
	sensitive=false,
	comment=[l]{\#},
	commentstyle=\color{purple}\ttfamily,
	stringstyle=\color{red}\ttfamily,
	morestring=[b]',
	morestring=[b]",
	showstringspaces=false
}


Si descrive nel seguito il procedimento seguito per creare una piccola demo del sistema di voto basata su Hyperledger Fabric. Prima di tutto si procederà a virtualizzare una semplice rete formata da un solo peer e un solo orderer, ed in seguito si descriverà l'uso di Hyperledger Composer per la configurazione del sistema fino alla creazione di opportune API REST su cui basare un'applicazione.

\section{Costruzione della rete}
	Si presuppone che nel sistema siano stati installati e configurati i software necessari, in particolare in questa sezione si uilizzeranno i binari di Hyperledger Fabric (che devono essere inseriti nel PATH), Docker e Docker-Compose.
	\subsection{Crypto Generator}
		Per prima cosa si genera il materiale crittografico necessario all'identificazione delle diverse entità in rete. Si tratta sostanzialmente di coppie di chiavi pubblica e privata, per consentiranno le operazioni di firma e verifica che avverranno quando i diversi attori effettueranno transazioni e comunicheranno tra loro.
		
		Cryptogen fa riferimento ad un file, \lstinline{crypto-config.yaml}, che contiene le informazioni sulla topologia di rete necessarie ad individuare le entità coinvolte.
		
		\lstinputlisting[basicstyle=\scriptsize\ttfamily, language=docker-compose]{chapters/code/hlfvoting/voting/crypto-config.yaml}
		
		Posizionandosi nella cartella in cui si trova questo file, il comando da lanciare è il seguente:
\begin{lstlisting}
cryptogen generate --config=./crypto-config.yaml
\end{lstlisting}
		Al termine gli artefatti generati si troveranno nella cartella \lstinline{crypto-config}.
		
	\subsection{Configuration Transaction Generator}
		Il tool \lstinline{configtxgen} si usa per la creazione di tre categorie di artefatti: 
		\begin{itemize}
			\item il \emph{genesis block} dell'orderer,
			\item le \emph{transazioni di configurazione} dei canali,
			\item le transazioni che definiscono gli \emph{anchor peer} - una per ogni organizzazione.
		\end{itemize}
		 
		Configtxgen sfrutta il file \lstinline{configtx.yaml}, che si troverà nel percorso indicato dalla variabile \lstinline{FABRIC_CFG_PATH}.
		
		Il contenuto del file sarà il seguente:
		\lstinputlisting[basicstyle=\scriptsize\ttfamily, language=docker-compose]{chapters/code/hlfvoting/voting/configtx.yaml}
		
		Nel caso specifico, risulta superfluo definire gli anchor peer in quanto di peer ne sarà presente solamente uno. Ci si limiterà quindi a dare i comandi necessari alla generazione del genesis block e della transazione di configurazione del canale, ovvero:
		
\begin{lstlisting}[language=bash]
export FABRIC_CFG_PATH=$PWD

configtxgen -profile VotingOrdererGenesis \
			-outputBlock ./voting-genesis.block
			
configtxgen -profile VotingChannel \
			-outputCreateChannelTx ./voting-channel.tx \
			-channelID votingchannel
\end{lstlisting}

	\subsection{Avvio della rete}
		Terminata la creazione degli artefatti necessari all'avvio della rete, si può procedere ad avviare i container docker come configurati nel file \lstinline{docker-compose.yml} riportato di seguito:
		\lstinputlisting[basicstyle=\scriptsize\ttfamily, language=docker-compose]{chapters/code/hlfvoting/voting/docker-compose.yml}
		
		I comandi da impartire sono i seguenti:
		
\begin{lstlisting}[language=bash]
# set useful variables
ARCH=`uname -m`
DIR="$( cd "$( dirname "${BASH_SOURCE[0]}" )" && pwd )"

ARCH=$ARCH docker-compose \
			-f "${DIR}"/voting/docker-compose.yml up -d
\end{lstlisting}

		Al termine della creazione dei container Docker è possibile impartire direttamente al peer i comandi per la creazione effettiva del canale e la registrazione del peer stesso nel canale appena creato:

\begin{lstlisting}[language=bash]
docker exec peer0.Stato.eurasia.com peer channel create \
			-o orderer.eurasia.com:7050 \
			-c votingchannel \
			-f /etc/hyperledger/configtx/voting-channel.tx

docker exec \
-e "CORE_PEER_MSPCONFIGPATH=/etc/hyperledger/msp/users/Admin@Stato.eurasia.com/msp" \
peer0.Stato.eurasia.com peer channel join -b votingchannel.block
\end{lstlisting}
	
\section{Utilizzo del composer}
	\subsection{Creare una nuova struttura di rete}
		Il concetto chiave nell'utilizzo di Hyperledger Composer è la \emph{``business network definition" (BND)}, che definisce il modello dei dati, le regole di accesso e la logica delle transazioni della nostra rete. Per creare la scheletro del progetto si può usare Yeoman generator, installato in precedenza assieme al Composer:
\begin{lstlisting}
yo hyperledger-composer:businessnetwork
\end{lstlisting}
		Inseriamo ``voting-network" come nome della rete e ``eurasia.voting" come namespace.
	\subsection{Definire le caratteristiche della rete}
		La nostra rete si fonda su asset (e.g i raggruppamenti di votanti), partecipanti (e.g votanti e candidati), transazioni come quella che permette ad un utente di votare e query che consentano la consultazione agevole dei dati. Come output del comando precedente, c'è un file di modello (\lstinline{.cto}) che conterrà le definizioni di ciascuna classe di asset, transazioni, partecipanti ed eventi. Il contenuto di questo file sarà:
		\lstinputlisting[basicstyle=\scriptsize\ttfamily,language=JavaScript]{chapters/code/voting-network/models/eurasia.voting.cto}

	\subsection{Scrivere la logica delle transazioni in JavaScript}
		Nel file di modello è stata definita una transazione \lstinline{Vote}, il cui scopo è quello di creare un nuovo voto ed indirizzarlo al candidato prescelto. Editiamo il file \lstinline{logic.js} per specificare il comportamento di questa transazione usando il linguaggio JavaScript, prestando attenzione ad utilizzare la sintassi accettata dal Transaction processor di Composer.
		\lstinputlisting[basicstyle=\scriptsize\ttfamily,language=JavaScript]{chapters/code/voting-network/lib/logic.js}

	\subsection{Aggiungere il controllo agli accessi}
		Creiamo il file \lstinline{permissions.acl} nella cartella \lstinline{voting-network}, e al suo interno specifichiamo le regole per il controllo degli accessi in questo modo:
		\lstinputlisting[basicstyle=\scriptsize\ttfamily,language=JavaScript]{chapters/code/voting-network/permissions.acl}

	\subsection{Generare il Business network archive}
		Terminata la definizione delle proprietà della rete è necessario impacchettarla in un file \lstinline{.bna} che rappresenta un archivio esportabile da cui poter poi effettuare il deploy su un sottostante sistema Fabric.
		Per fare ciò è necessario:
		\begin{itemize}
			\item Spostarsi nella cartella \lstinline{voting-network}
			\item Lanciare il seguente comando da terminale:
				\begin{lstlisting}
composer archive create -t dir -n .
				\end{lstlisting}
		\end{itemize}
		Al termine dell'esecuzione potremo trovare un file chiamato \lstinline{voting-network@0.0.1.bna} nella cartella da cui abbiamo lanciato il comando.
		
\section{Deploy della rete sul sottosistema Fabric}