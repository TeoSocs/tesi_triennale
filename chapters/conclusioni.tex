\section{Valutazione retrospettiva}
	L'esperienza di stage è stata assolutamente positiva. L'argomento si è rivelato particolarmente complesso da comprendere, ma l'approccio adottato, il supporto e gli stimoli ricevuti sono stati più che adeguati e mi hanno permesso un percorso di crescita estremamente soddisfacente. La natura interdisciplinare del progetto mi ha portato ad approfondire argomenti accennati ma non particolarmente approfonditi nel corso degli studi per la laurea triennale quali tecnologie di rete, sicurezza e crittografia. Inoltre, la mancanza di testi universalmente adottati come riferimento mi ha portato a consultare una gran quantità di materiale fino ad arrivare ad affrontare direttamente le pubblicazioni scientifiche sull'argomento, cosa che si è rivelata estremamente interessante e che immagino si rivelerà molto utile in futuro. A corollario, ho trovato piacevoli spunti di riflessione anche nell'osservare più linguaggi di programmazione diversi da quelli affrontati nel corso di laurea (tra cui, ma non solo, Go e Solidity) e nel confronto tra loro e quanto già imparato finora.
	
	Sono consapevole della fortuna avuta nel poter confrontarmi con argomenti tanto attuali quanto stimolanti nel contesto di un'attività di tirocinio, ringrazio moltissimo i miei referenti interni all'azienda Enrico Notariale e Luca Brizzolari per le innumerevoli opportunità di confronto offertemi e mi auguro di poter proseguire i miei studi sull'argomento attraverso ulteriori esperienze in azienda e il corso di laurea magistrale.
	
	\subsection{Raggiungimento degli obiettivi}
		Nel piano di lavoro erano stati individuati i seguenti obiettivi:
		\begin{itemize}
			\item \textbf{Minimi:}
				\begin{itemize}
					\item \textbf{min01}: Analisi delle potenzialità della tecnologia blockchain, in particolare Hyperledger Fabric;
					\item \textbf{min02}: Installazione dell’architettura in un ambiente di sviluppo;
					\item \textbf{min03}: Creazione di un ambiente di sviluppo.
				\end{itemize}
			
			\item \textbf{Massimi}:
				\begin{itemize}
					\item \textbf{max01}: SWOT Analysis;
					\item \textbf{max02}: Predisposizione di una demo.
				\end{itemize}
			
			\item \textbf{Formativi}:
				\begin{itemize}
					\item \textbf{for01}: Acquisizione abilità nella gestione di aspetti riguardanti l’analisi di un progetto innovativo in un ambito aziendale (e.g. nuovi linguaggi, sistemi);
					\item \textbf{for02}: Interagire con un progetto Open-Source e la relativa community;
					\item \textbf{for03}: Sperimentare tecnologie interessanti per l’ingresso nel mondo del lavoro.
				\end{itemize}
		\end{itemize}
		
		Nel complesso posso dire raggiunti gli obiettivi pianificati, quasi tutti rispettando in pieno le aspettative. L'unico punto in cui è stato necessario scendere a compromessi è stata la predisposizione della demo: sebbene sia stata progettata e sia stato predisposto un accenno di implementazione, questa non può certo dirsi completa. La mutevolezza della piattaforma su cui stavamo lavorando si è resa evidente con la presenza di file nel repository non ancora documentati, e cambiamenti considerevoli da una settimana all'altra. Di conseguenza, si è ritenuto più proficuo indugiare nelle attività di analisi e confronto tra tecnologie riservando a progetti futuri l'implementazione effettiva del prototipo.
		
\section{Conclusioni e prosieguo del progetto}
	In conclusione, l'esperienza di tirocinio ha confermato quanto suggerito dagli studi di Gartner: blockchain è una realtà tanto dirompente quanto delicata da implementare in ambito business. Soffre dei problemi di gioventù tipici di un sistema innovativo, a cui vanno ad aggiungersi criticità specifiche dovute alla sua complessità e al suo orientamento a sistema certificatore e garante, che non consente incertezze di rischio nel suo utilizzo. Nonostante ciò, promette innovazioni troppo importanti per essere ignorata. Ciò suggerisce una riflessione alle aziende più attente all'argomento: conviene approfittare di questo periodo di incubazione per studiare, sperimentare e accumulare le conoscenze necessarie per essere operativi e consapevoli nel seguire l'evoluzione tecnologica dei prossimi anni. 
	
	In accordo con i miei referenti, nel prossimo periodo continueranno i miei rapporti con Infocamere per proseguire ed integrare quanto fatto finora consolidando le conoscenze acquisite e arrivando sperabilmente ad una implementazione efficace di una o più sperimentazioni. Sono convinto che questo percorso possa portare l'azienda ad acquisire quella confidenza necessaria per perseguire i propri obiettivi di ricerca e innovazione, e possa portare a me una crescita inestimabile dal punto di vista intellettuale e professionale che spero di riuscire a ricambiare con risultati adeguati.
	
	
	
	
	
	
	
	
	
	
	
	
	
	
	
	
	
	
	
	
	
	