In questo capitolo si analizzano le scelte architetturali fatte argomentandole, concludendo con delle possibilità di miglioramento che si potranno presentare con l'avanzamento dello scenario tecnologico attuale.

\section{Qualità del voto}
	Il voto è definito dall'art. 48 \cite{art48} della Costituzione Italiana come "personale ed eguale, libero e segreto".
	\subsection{Personalità}\label{subsec:personalita_voto}
		La personalità del voto, intesa come la necessità di esercitarlo di persona e non tramite terzi, è delegata in questa proposta al servizio di sign-on. Al momento attuale, è possibile fare affidamento su diversi sistemi di autenticazione: per il corretto funzionamento del sistema è previsto che ogni utente abbia una coppia di chiavi pubblica e privata distribuita precedentemente, sotto verifica di un operatore. Ad esempio, è possibile distribuire queste chiavi contestualmente alla consegna della tessera elettorale. La natura certificata di queste chiavi le rende ideali come metodo di autenticazione ma è possibile aggiungere ulteriori controlli come una verifica tramite dati biometrici.

	\subsection{Eguaglianza}
		L'eguaglianza del voto consiste nell'avere ogni voto lo stesso valore di tutti gli altri. Si articola nell'impedire, in particolare, il voto plurimo e il voto multiplo. Nel sistema proposto, il voto plurimo è scongiurato dal codice del chaincode della Vote chain, il quale permette solo incrementi unitari al saldo dei candidati. Uno scenario di voto multiplo è invece controllato dalla Signature chain e dalla struttura del client: è infatti indispensabile che la procedura di voto sia strettamente successiva alla procedura di firma in SC, e non invocabile in maniera indipendente. In tal modo è possibile garantire che una stessa persona non reiteri l'esecuzione del chaincode di voto. Pur ammettendo il caso in cui un attaccante riesca a violare il client, questo potrà reiterare voti firmati dalla stessa ring signature: un controllo automatico può facilmente accorgersi della non coerenza del numero di votanti nel raggruppamento e il numero di voti espressi dal raggruppamento, invalidando di fatto le transazioni ma mantenendo confinato l'evento disastroso: si possono applicare le procedure invocate attualmente qualora venissero riscontrate irregolarità che compromettano la validità dei voti di una sezione elettorale.

	\subsection{Libertà}\label{subsec:liberta_voto}
		La libertà del voto viene garantita dal negare la possibilità di voto qualora il votante fosse stato costretto con mezzi illeciti a esprimere una certa preferenza. Nel sistema di voto attuale è garantita dal controllo degli operatori di seggio. Nei sistemi di voto remoto l'argomento risulta spinoso: è infatti estremamente difficile stabilire una situazione di particolare pressione dell'elettore, o l'interferenza di esterni. Nel caso di voto elettronico, forzare il client ufficiale a poter essere installato solo su dispositivi muniti di telecamera frontale potrebbe risolvere il problema. Il controllo può essere svolto da appositi addetti eventualmente aiutati da un'intelligenza artificiale simile a quella utilizzata da Unilever \cite{ai_unilever} per i colloqui di lavoro. Questo non scongiura ogni possibile manipolazione, ma va evidenziato che il controllo sul sistema proposto non lo rende meno sicuro del sistema già in essere: permette anzi un livello di garanzia estremamente più alto di quello dei sistemi di voto per corrispondenza, già accettati in Italia in caso di votanti dall'estero e adottato in maniera diffusa in Svizzera.

	\subsection{Segretezza}
		La segretezza del voto nel sistema proposto ha come fulcro la completa separazione delle due blockchain. Affidando al client il compito di mantenere la continuità dell'operazione di voto, l'accoppiamento tra votante e voto espresso non può essere ricostruito con certezza in alcun modo. L'unica informazione che può trapelare riguarda il momento in cui viene registrata la votazione nelle due chain, tuttavia questa informazione è molto meno affidabile di quanto possa sembrare a causa della \href{sec:teorema_CAP}{consistenza eventuale} della blockchain. Sebbene qualche informazione trapeli, l'elevato numero di transazioni contemporanee previsto e l'incertezza nel determinare l'esatto momento di voto rende di fatto inservibili queste informazioni.


\section{Necessità dei raggruppamenti}
	La suddivisione in sezioni elettorali è assolutamente necessaria nel sistema di voto attualmente adottato a causa di problemi logistici: è impensabile infatti raggruppare tutte le schede e svolgere un'unica operazione di scrutinio, soprattutto considerando il livello di sorveglianza che sarebbe richiesto in ogni momento dell'operazione. In un sistema automatico come quello proposto, esente da queste problematiche, sono stati introdotti raggruppamenti di elettori per differenti ragioni. In primo luogo, la complessità computazionale degli algoritmi più noti e affermati per la ring signature è lineare: questo preclude l'uso significativo di questa tecnologia (centrale nel modello proposto) per insiemi di chiavi sufficientemente grandi. In secondo luogo, nel caso un attacco permettesse di replicare lo stesso voto più volte la divisione in raggruppamenti permette di contenere i danni al solo raggruppamento interessato.
	È da sottolineare come, nel caso di un sistema elettronico, l'indipendenza dai problemi logistici dei sistemi tradizionali permetta di variare di volta in volta la composizione dei raggruppamenti, rendendo inutile qualsiasi informazione trapelata dalle votazioni precedenti. Questa è una condizione certamente più sicura di quella attuale, dove una semplice osservazione sul posto permette di identificare gli appartenenti a ciascuna sezione e far trapelare quindi informazioni (per quanto parziali) sulle preferenze di voto espresse: essendo i risultati delle votazioni per seggio pubblici, il trapelare di qualsiasi informazione sulla composizione del seggio è da considerarsi quantomeno indesiderato.

\section{Prevedibili miglioramenti futuri al sistema}
	Al momento passaggi critici come l'autenticazione utente e il passaggio dalla SC alla VC devono essere svolti off-chain, ma future scoperte potrebbero aprire la strada verso implementazioni più complete.
	Portare l'autenticazione utente su Blockchain è un progetto che si lega con la creazione di un sistema di identità digitale decentralizzata. Immaginando un futuro (ad ora remoto) in cui siano disponibili documenti affidabili on-chain attraverso qualche tecnologia, l'autenticazione utente potrebbe svolgersi in questo modo guadagnando le caratteristiche di decentralità e sicurezza proprie della Blockchain. \\
	Più vicino alla situazione attuale è un meccanismo che permetta di abilitare alla votazione su VC direttamente da SC, senza rinunciare all'anonimato del votante. Al momento, ogni chaincode di Hyperledger è eseguito in un ambiente completamente isolato per ragioni di sicurezza. Sono già previste successive implementazioni che permettano a diversi chaincode di interagire tra di loro. Questo aprirebbe ad una modifica del sistema di voto proposto: sarebbe possibile implementare su VC un sistema di token: questi token sarebbero generati quando viene indetta la votazione in base al numero di aventi diritto al voto, e versati dai chaincode verso dei ``portafogli" temporanei da loro generati casualmente. Questi sarebbero poi passati al client già cifrati in maniera da mantenere chiunque, eccetto il software del chaincode, all'oscuro dell'accoppiamento votante-portafoglio. In questo modo sarebbe possibile prescindere dai raggruppamenti affidando la sicurezza del sistema unicamente alla solidità della Blockchain. L'aspetto negativo è che si perderebbe la capacità dei raggruppamenti di confinare un eventuale attaccante.