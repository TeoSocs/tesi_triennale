\section{blockchain}
	L'invenzione di Bitcoin nel 2008 ha portato alla ribalta un nuovo concetto, quello di blockchain, fondamentale per la sua realizzazione ma non limitato ad essa. Inizialmente confusa dai più con la moneta elettronica che le ha dato notorietà,  blockchain si è affermata come idea a sè stante, che promette di influenzare pesantemente i paradigmi a cui siamo abituati nel campo della condivisione di informazioni, della finanza, della sicurezza informatica e molti altri. Per alcuni, si assiste alla nascita di una nuova \emph{Internet del Valore}. Per altri possiamo parlare di \emph{Democrazia digitale}, vista come la trasposizione informatica di concetti quali decentralizzazione, trasparenza, sicurezza e immutabilità uniti alla gestione completamente nuova della fiducia all'interno di una rete informatica.
	\paragraph{Percezione attuale}: Secondo Gartner, Inc.\cite{gartner}, blockchain si trova al momento nel \emph{picco delle aspettative esagerate}. L'entusiasmo è ai limiti dell'eccessivo anche in rapporto all'effettiva natura dirompente del fenomeno, tuttavia valutandone le potenzialità a lungo termine c'è la convinzione che questa tecnologia darà forma ad un nuovo concetto di economia, intesa come ecosistema commerciale peer-to-peer e many-to-many, e comportando la scomparsa del modello controllato in maniera centralizzata presente oggi. Si ritiene che questa innovazione porterà ad una rivalutazione forzata dei portfoli tecnologici, delle strutture organizzative, delle pratiche di gestione dei rischi e dei modelli di business. Allo stesso modo, si parla di rivalutazione forzata anche sulle tematiche riguardanti pratiche legali, fiscali e contabili, interazione tra società e diritti di proprietà individuale.
	\paragraph{Le critiche}: Tanto entusiasmo attorno a questa idea rivoluzionaria è smorzato dai notevoli compromessi da essa richiesti, e dalla cautela necessaria nel valutare con attenzione un cambio di paradigma tanto pesante all'interno di sistemi critici, in cui i concetti e i protocolli adottati sono ormai rodati e assodati. Sempre Gartner evidenzia come blockchain, e in generale i ledger distribuiti, siano presentati come \emph{``La risposta"} ad ogni tipo di processo, modello operativo e tecnologia preesistente. Tuttavia, molti fattori continuano ad inibirne l'adozione su larga scala, tra cui il dissenso sull'applicabilità delle diverse tipologie di blockchain, l'adeguatezza dei meccanismi di consenso, la mancanza di standard, il caotico scenario di startup e tecnologie ``\emph{core}" di efficacia e sicurezza ancora da provare. Inoltre, la mancanza di framework robusti mette in discussione l'effettiva opportunità di investimenti eccessivamente vincolanti, specie se esistono alternative più affermate, conosciute e testate.

\section{L'azienda}
	Lo stage si è svolto presso Infocamere S.C.p.A, nella sua sede di Padova.
	In qualità di società in-house delle Camere di Commercio italiane, Infocamere mette a disposizione degli enti camerali le proprie competenze sul fronte dell’organizzazione e della gestione sempre più efficiente dei processi interni, sviluppando servizi informatici basati su tecnologie ad elevato standard qualitativo a supporto delle numerose attività di back office delle Camere, in chiave di semplificazione. Il corretto funzionamento di queste attività all'interno del Sistema Camerale è infatti determinante per garantire la qualità dei dati e dei servizi che le Camere di Commercio offrono a imprese, Pubbliche Amministrazioni e professionisti.
	
	Questa azione pone al centro l’obiettivo di dematerializzare e integrare fra loro i flussi informativi e si riflette in tutti i servizi che la Società offre a supporto delle attività delle Camere di Commercio. Tra questi rientrano gli strumenti di gestione del Registro delle Imprese, grazie ai quali le Camere di Commercio possono governare i flussi operativi relativi a specifiche competenze di legge, quali: ambiente e agricoltura, commercio estero e contributi alle imprese, regolazione del mercato, gestione di albi e ruoli abilitanti. A questi servizi InfoCamere affianca l’offerta di una serie di servizi amministrativi gestionali evoluti, tra cui quelli per la gestione della contabilità e del personale, per la pianificazione strategica e il controllo di gestione, per il monitoraggio e l'alimentazione della banca dati del diritto annuo dovuto dalle imprese.

\section{Organizzazione dei contenuti}
	Nei prossimi capitoli verrà presentata la nascita della tecnologia blockchain, partendo dai problemi che si promette di risolvere e arrivando poi a descrivere la sua prima applicazione pratica in Bitcoin (\hyperref[cap:origini]{Cap. \ref*{cap:origini}}); si proseguirà quindi con una definizione più generale di blockchain e con una analisi ad alto livello della tecnologia e della sua applicabilità (\hyperref[cap:analisiTecnologia]{Cap. \ref*{cap:analisiTecnologia}}). \\
	Si presenterà in seguito una proposta di use case, utilizzata nel contesto di stage per provare con mano quanto ipotizzato in fase di analisi (\hyperref[cap:proposta]{Cap. \ref*{cap:proposta}}), fornendo poi una descrizione dello scenario attuale, con accenno agli ambiti di ricerca più promettenti e uno sguardo alle possibilità future della tecnologia (\hyperref[cap:scenario]{Cap. \ref*{cap:scenario}}). \\
	Si passerà quindi alle conclusioni, confrontando gli obiettivi prefissati con quelli effettivamente raggiunti e valutando possibili nuovi progetti in continuità con quanto svolto finora (\hyperref[cap:conclusioni]{Cap. \ref*{cap:conclusioni}}).