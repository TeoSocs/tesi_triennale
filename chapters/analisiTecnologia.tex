\section{Generalizzazione}
	In seguito all'esordio di Blockchain assieme a Bitcoin sono state presentate numerose proposte per sviluppi ed implementazioni della tecnologia. Nel seguito di questa sezione si presenta quello che vuole essere uno schema delle caratteristiche salienti di un sistema basato su Blockchain: sono riportati le principali scelte di progettazione che lo caratterizzano e alcune delle soluzioni ad oggi adottate. La ricerca è estremamente attiva a riguardo, perciò questo elenco non sarà e non vuole essere esaustivo: lo scopo è quello di chiarire a che tipologie di sistema si fa riferimento parlando genericamente di \emph{Blockchain}.
	\subsection{Definizione tecnica}
		Una buona definizione generica di Blockchain è quella proposta da Imran Bashir in \emph{Mastering Blockchain} \cite{mastering_blockchain}: essenzialmente si tratta di un registro distribuito peer-to-peer, crittograficamente sicuro, append-only, immutabile (o meglio, estremamente difficile da modificare) e aggiornabile solo previo consenso da parte dei peer. Esistono ovviamente molte altre definizioni, ciascuna con enfasi diversa sui vari aspetti di Blockchain, tuttavia l'estrema sintesi di quella di Bashir permette di concentrarsi sugli aspetti imprescindibili che la caratterizzano, lasciando alle scelte di progettazione successive il compito di determinarne i particolari.
		
	\subsection{Accesso}
		La prima grande suddivisione in macro categorie dei sistemi basati su Blockchain ne riguarda le modalità di accesso. La Blockchain può essere:
		\begin{itemize}
			\item \textbf{Pubblica} nel momento in cui il sistema è aperto al pubblico e chiunque può installarne un client, scaricare il registro e partecipare attivamente ai processi di validazione dei blocchi. Tipicamente ogni utente scarica e mantiene una copia locale del registro e tramite un sistema di consenso distribuito si raggiunge un accordo con il resto della rete sull'effettivo stato ``ufficiale" dei dati. Tipicamente si affidano ad algoritmi di consenso basati su proof-of-work o proof-of-stake. Sono anche dette \emph{permissionless ledgers}, e l'esempio più noto è Bitcoin;
			\item \textbf{Privata} nel caso l'accesso sia ristretto ad un limitato insieme di utenti o organizzazioni che abbiano deciso di condividere dei registri comuni. In questa accezione di Blockchain acquista importanza l'autenticazione degli utenti, e si ottiene una maggiore riservatezza dei dati a costo di rinunciare alla natura completamente ``\emph{trustless}" delle blockchain pubbliche. Ciò porta a differenze sostanziali nella gestione del consenso: è possibile prescindere da prove onerose come la proof-of-work, e soprattutto viene a mancare la necessità di rendere vantaggioso a ciascun peer il mantenimento del sistema.
			Il grado di chiusura di questo tipo di blockchain può variare anche molto: è possibile ad esempio permettere a ciascun utente di osservare il registro ma limitarne ad un insieme ristretto la modifica. Sono anche dette \emph{permissioned ledgers} e esempi possibili sono Ripple e i sistemi costruiti su Hyperledger.
		\end{itemize}
	
	\subsection{Consenso}
		La tipologa di consenso adottata caratterizza fortemente l'intero sistema Blockchain, ed è strettamente legata alla tipologia di accesso scelta.
		\begin{itemize}
			\item \textbf{Proof of work}: è il meccanismo di consenso adottato da Bitcoin. Richiede la soluzione ad un problema matematico computazionalmente impegnativo (e.g. \emph{hashcash}) come prova delle risorse investite per far accettare la propria proposta dalla rete;
			\item \textbf{Proof of stake}: è il meccanismo di consenso usato da Peercoin (e si parla di adottarlo prossimamente in Ethereum \cite{casper}). Sbilancia la probabilità di far accettare una proposta di cambiamento del registro in favore dei nodi di maggiore ``importanza" all'interno della rete. In una Blockchain che rappresenti una valuta, ad esempio, l'importanza consiste nella quantità di valuta posseduta. Nella variante dei consensi \emph{deposit-based} invece si acquista importanza versando quella che è in pratica una caparra. L'idea di base è che un attaccante dovrebbe investire talmente tante risorse nell'oggetto del suo attacco da renderlo non profittevole, in quanto le ricadute sul suo investimento sarebbero troppo pesanti. Una variante è la \emph{Delegated Proof of Stake} in cui i nodi importanti delegano la validazione di ciascuna transazione tramite votazione;
			\item \textbf{Proof of elapsed time}: è un meccanismo di consenso introdotto da Intel \cite{poet} allo scopo di limitare l'enorme dispendio di energie dei sistemi proof-of-work. Utilizza hardware proprietario certificato per garantire l'affidabilità di un timer, ottenendo di fatto lo stesso effetto dei PoW con irrisorio consumo di energia;
			\item \textbf{Reputation based}: è una variante dei sistemi proof-of-stake dal nome autoesplicativo. Il leader viene eletto sulla base della reputazione che si è costruito nel tempo all'interno della rete, che può basarsi su lavoro effettuato o su espressione degli altri nodi;
			\item \textbf{Federated (Byzantine) agreement}: è un sistema non-trustless, e pertanto adatto a blockchain di tipo permissioned. I nodi tengono traccia di un gruppo di peer pubblicamente riconosciuti come affidabili e propaga solo le transazioni validate dalla maggioranza dei nodi affidabili. È implementato nello Stellar Consensus Protocol \cite{stellar_protocol}; 
			\item \textbf{Practical Byzantine Fault Tolerance}: è la soluzione originale al problema dei generali bizantini proposta da Castro e Liskov nel 1999 \cite{PBFT}.
		\end{itemize}

	\subsection{Struttura}
		La scelta è in questo caso tra:
		\begin{itemize}
			\item Tokenized blockchain;
			\item Tokenless blockchain.
		\end{itemize}
		Blockchain è nata come supporto ad un sistema di scambio di token, e le applicazioni al momento più diffuse rispecchiano questo legame nei confronti di un'unità minima di informazione trasferibile. Soprattutto nel caso di blockchain permissioned, però, è possibile prescindere dal concetto di token, condividendo informazioni analogamente a quello che si può fare mediante un database tradizionale. Hyperledger permette di implementare sistemi di questo tipo, basati su un database condiviso (lo \emph{state}) in cui ogni modifica è registrata in una blockchain che funge da "log certificato".
	
	\subsection{Indipendenza}
		Un ultima distinzione degna di nota è quella tra blockchain stand-alone e sidechain \cite{sidechain}. Finora, le implementazioni di sidechain riguardano esclusivamente le tokenized blockchain. Una sidechain si appoggia ad una stand-alone e permette i trasferimenti di token dall'una all'altra. Si appoggia su varianti di proof-of-stake in cui è necessario impegnare dei coin di una catena per generarne nell'altra. I trasferimenti possono essere unidirezionali o bidirezionali. L'interesse sulle sidechain è estremamente alto poiché queste abilitano la creazione di monete che si appoggiano a criptovalute diffuse come Bitcoin e ne vanno a colmare delle lacune. Interessante a proposito il punto di vista di Ferdinando Ametrano sull'uso di sidechain per ottenere la stabilità dei prezzi con una criptovaluta, riportata nel suo paper del 2014 ``\emph{Hayek Money: the Cryptocurrency Price Stability Solution}" \cite{hayek_money}.

	\subsection{Teorema CAP e Blockchain}
		Può sembrare che le varie implementazioni di Blockchain illustrate violino il \href{sec:teorema_CAP}{teorema CAP} presentato in \ref{sec:teorema_CAP}. Ciò è inesatto: in particolare Blockchain persegue disponibilità e tolleranza di partizione in maniera istantanea, mentre la coerenza è sacrificata e ottenuta solo attraverso un lasso di tempo. Si parla in questo caso di \emph{coerenza eventuale}, ed è il motivo per cui il protocollo Bitcoin è stato di fatto artificialmente rallentato mediante un problema a difficoltà dinamica, che assicurasse un impegno per la proof-of-work di circa 10 minuti. Un qualsiasi sistema Blockchain può dirsi coerente solo facendo riferimento a transazioni con una certa ``età", che siano state accettate e validate col tempo dai vari peer.

\section{Benefici e limitazioni}
	\emph{Spunto di riflessione: blockchain innovazione sia in ambito economico che in ambito informatico: rivoluziona nel primo caso, innova nel secondo. \url{https://www.linkedin.com/pulse/ending-bitcoin-vs-blockchain-debate-gideon-greenspan}}