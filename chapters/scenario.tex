\section*{DISCLAIMER} \emph{Questa sezione è un cantiere aperto. NON è definitiva in nessuna sua parte, NON nei contenuti, NON nella struttura, NON nel tono generale del discorso. Si tratta di una serie di appunti che sto prendendo e che saranno la base su cui scriverò poi la versione definitiva, li inserisco per mantenerne traccia e per avere ben chiara la direzione che sta intraprendendo il discorso.}

\section{Innovazioni vicine}
    Esistono già:
    \subsection{Criptomonete}
        \begin{itemize}
            \item \textbf{Ripple}: è il più famoso sistema di scambio di valuta basato su blockchain permissioned, adottato tra gli altri da MUFG, RBC, Santander, Unicredit, BBVA \url(https://ripple.com/);
            \item \textbf{Litecoin}: è una criptomoneta molto simile a Bitcoin da cui differisce per alcune caratteristiche chiave, su tutte il tempo necessario all'elaborazione di un blocco (2 minuti e mezzo contro i 10 minuti di Bitcoin) e il sistema di consenso. Litecoin infatti usa \emph{scrypt} per la sua proof-of-work, una funzione gravosa sulla memoria piuttosto che sul processore che punta a evita il predominio delle server farm con hardware dedicato che controllano le operazioni di mining di Bitcoin;
            \item \textbf{Peercoin}: è una criptomoneta alternativa che adotta un algoritmo di consenso ibrido tra proof-of-work e proof-of-stake, allo scopo di portare un'alternativa solida all'enorme consumo energetico di Bitcoin;
            \item \textbf{Monero}: fork di Bitcoin che utilizza CryptoNote, un protocollo basato sulla Ring Signature orientato a rafforzare l'anonimato nella blockchain;
            \item \textbf{ZCash e ZCoin}: come Monero si pongono l'obiettivo di garantire la privacy in un sistema Blockchain, usano algoritmi di tipo zero-knowledge sebbene con piccole differenze tra di loro \cite{zcoin_vs_zcash};
            \item \textbf{CoinJoin}: metodo di anonimizzazione per le transazioni Bitcoin;
            \item crowdfunding su blockchain
        \end{itemize}
     
    \subsection{Non-Criptomonete}
        \begin{itemize}
            \item \textbf{Ethereum}
            \item \textbf{OpenTimestamps} servizio di timestamping che si appoggia alla blockchain di Bitcoin \url{https://opentimestamps.org/};
            \item \textbf{Namecoin}: servizio basato su blockchain che si propone di potenziare decentralizzazione, sicurezza, resistenza alla censura, privacy e velocità di componenti alcune dell'infrastruttura di Internet come i DNS;
            \item \textbf{Blockchain specifiche}: Chain ad-hoc con utilizzo ben preciso. \\
                Esempi: \emph{Everledger} per diamanti e beni di lusso, \emph{Filament} per reti wireless sicure in IoT;
            \item \textbf{Blockchain a livello enterprise}: Devono rispettare requisiti per testing, documentazione, integrazione, sicurezza. \\
                Esempi: \emph{Bloq, Chain, Hyperledger}
            \item \textbf{Hawk}: The Blockchain Model of Cryptography and Privacy-Preserving Smart Contracts \url{http://ieeexplore.ieee.org/document/7546538/?reload=true}
        \end{itemize}

\section{Sfide e ambiti di ricerca}
    \subsection{Efficienza e scalabilità}
        \begin{itemize}
            \item Problemi crittografici nuovi
            \item Algoritmi di consenso
        \end{itemize}
    \subsection{Standardizzazione e interoperabilità}
    \subsection{Privacy}
        \subsubsection{Algoritmi zero-knowledge}
            \begin{itemize}
                \item \textbf{Zero-Knowledge Password Proof} \\
                    \url{https://en.wikipedia.org/wiki/Zero-knowledge_password_proof}
                \item \textbf{Piattaforma Enigma} \\
                        \url{https://www.enigma.co/enigma_full.pdf}
                \item \textbf{zk-SNARK} \\
                    \url{https://z.cash/technology/zksnarks.html}
            \end{itemize}
        \subsubsection{Obfuscation}
            vedi pag 244 di Understanding Bitcoin

\section{Cosa si potrà fare}
    \begin{itemize}
        \item Reti IoT sicure basate su Blockchain condivise
        \item Condivisione dati sensibili (cartelle cliniche?)
        \item Digital Identity
        \item Immigrazione e controlli doganali
        \item Cronotachigrafi
        \item Gestione DRM (digital rights management)
        \item Digital assets e smart properties
        \item Moneta basata su Bitcoin per transazioni efficaci tra privati
\end{itemize}
