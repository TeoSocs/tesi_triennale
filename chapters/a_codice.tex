\definecolor{mygreen}{rgb}{0,0.6,0}
\definecolor{mygray}{rgb}{0.5,0.5,0.5}
\definecolor{mymauve}{rgb}{0.58,0,0.82}

\lstset{
	basicstyle=\footnotesize \ttfamily,
	commentstyle=\color{mygreen},
	keywordstyle=\color{blue},
	stringstyle=\color{mymauve},
	numbers=none,
	%numberstyle=\tiny\color{lightgray},
	%numbersep=5pt,
	%frame=single,
	rulecolor=\color{mygray}, 
	breakatwhitespace=false,         % sets if automatic breaks should only happen at whitespace
	breaklines=true,                 % sets automatic line breaking
	tabsize=4
}

\definecolor{darkgray}{rgb}{.4,.4,.4}
\definecolor{purple}{rgb}{0.65, 0.12, 0.82}

%define Javascript language
\lstdefinelanguage{JavaScript}{
	keywords={typeof, new, true, false, catch, function, return, null, catch, switch, var, if, in, while, do, else, case, break},
	keywordstyle=\color{blue}\bfseries,
	ndkeywords={class, export, boolean, throw, implements, import, this},
	ndkeywordstyle=\color{darkgray}\bfseries,
	identifierstyle=\color{black},
	sensitive=false,
	comment=[l]{//},
	morecomment=[s]{/*}{*/},
	commentstyle=\color{purple}\ttfamily,
	stringstyle=\color{red}\ttfamily,
	morestring=[b]',
	morestring=[b]"
}

\lstset{
	language=JavaScript,
	extendedchars=true,
	basicstyle=\footnotesize\ttfamily,
	showstringspaces=false,
	showspaces=false,
	numbers=none,
	numberstyle=\footnotesize,
	numbersep=9pt,
	tabsize=2,
	breaklines=true,
	showtabs=false,
	captionpos=b
}

%\section{Costruzione della rete}
%	\lstinputlisting[language=bash]{chapters/code/build.sh}
	
%\section{Configurazione crypto-tool}
%	\lstinputlisting{chapters/code/crypto-config.yaml}
	
%\section{Configurazione configtxgen}
%	\lstinputlisting{chapters/code/configtx.yaml}

%\section{Compose}
%	\lstinputlisting{chapters/code/docker-compose.yml}
	
\section{Utilizzo del composer}
	\subsection{Creare una nuova struttura di rete}
		Il concetto chiave nell'utilizzo di Hyperledger Composer è la \emph{``business network definition" (BND)}, che definisce il modello dei dati, le regole di accesso e la logica delle transazioni della nostra rete. Per creare la scheletro del progetto si può usare Yeoman generator, installato in precedenza assieme al Composer:
\begin{lstlisting}
yo hyperledger-composer:businessnetwork
\end{lstlisting}
		Inseriamo ``voting-network" come nome della rete e ``eurasia.voting" come namespace.
	\subsection{Definire le caratteristiche della rete}
		La nostra rete si fonda su asset (e.g i raggruppamenti di votanti), partecipanti (e.g votanti e candidati), transazioni come quella che permette ad un utente di votare e query che consentano la consultazione agevole dei dati. Come output del comando precedente, c'è un file di modello (\emph{.cto}) che conterrà le definizioni di ciascuna classe di asset, transazioni, partecipanti ed eventi. Il contenuto di questo file sarà:
		\lstinputlisting{chapters/code/voting-network/models/eurasia.voting.model.cto}

	\subsection{Scrivere la logica delle transazioni in JavaScript}
		Nel file di modello è stata definita una transazione \emph{Vote}, il cui scopo è quello di creare un nuovo voto ed indirizzarlo al candidato prescelto. Editiamo il file \emph{logic.js} per specificare il comportamento di questa transazione usando il linguaggio JavaScript, prestando attenzione ad utilizzare la sintassi accettata dal Transaction processor di Composer.
		\lstinputlisting{chapters/code/voting-network/lib/logic.js}

	\subsection{Aggiungere il controllo agli accessi}
		Creiamo il file \emph{permissions.acl} nella cartella \emph{voting-network}, e al suo interno specifichiamo le regole per il controllo degli accessi in questo modo:
		\lstinputlisting{chapters/code/voting-network/permissions.acl}

	\subsection{Generare il Business network archive}
		Terminata la definizione delle proprietà della rete è necessario impacchettarla in un file \emph{.bna} che rappresenta un archivio esportabile da cui poter poi effettuare il deploy su un sottostante sistema Fabric.
		Per fare ciò è necessario:
		\begin{itemize}
			\item Spostarsi nella cartella \emph{voting-network}
			\item Lanciare il seguente comando da terminale:
				\begin{lstlisting}
composer archive create -t dir -n .
				\end{lstlisting}
		\end{itemize}
		Al termine dell'esecuzione potremo trovare un file chiamato \emph{voting-network@0.0.1.bna} nella cartella da cui abbiamo lanciato il comando.